\chapter{\IfLanguageName{dutch}{Interviews met consultants}{Interviews with consultants}}%
\label{ch:interviews}

Het is belangrijk om te weten hoe RPA gebruikt wordt binnen delaware, hoe ver ze hier mee staan en hoe de verschillende consultants, de toekomstige gebruikers van de 'bots', staan tegenover deze technologie. Ook is het belangrijk om hun dagelijkse processen te begrijpenl en deze te bekijken in het licht van RPA. In dit hoofdstuk worden de interviews die afgenomen zijn met de verschillende consultants besproken.

\section{Ondervraagde consulants}
\label{sec:ondervraagde-consulants}

Voor de interviews werden 4 verschillende consultants ondervraagd. Deze consultants werden gekozen op basis van hun team en level binnen delaware. De verschillende consultants zijn:

\begin{itemize}
    \item Sam Debruyn: Senior Technical Consultant binnen het Automation team
    \item Gilles Styns: Technical Consultant binnen het SAP Development team
    \item Isabel De Bruyn: Senior Functional Consultant binnen het SAP Finance team
    \item Ian Kerkhove: Senior Technical Consultant binnen het SAP Development team
\end{itemize}

De 4 ondervraagden zijn dus een goede mix van ervaren en minder ervaren consultants, die uit verschillende teams komen. Hierdoor wordt een goed beeld van de kennis van RPA binnen de levels en teams gecreëerd.

Aangezien het niveau van RPA kennis bij de ondervraagden verschillend is, werden er 2 vragenlijsten opgesteld. De eerste vragenlijst werd allereerst gebruikt om de algemene kennis van RPA te testen en te kijken hoe de consulants die niet dagelijks bezig zijn met automatiseringen hier tegenover staan. Een tweede belangrijk punt van deze lijst was het vinden van processen die volgens de ondervraagden voordeel zouden kunnen halen uit RPA-automatisering. Deze vragenlijst werd gebruikt bij de ondervraging van Gilles Styns, Ian Kerkhove en Isabel De Bruyn.
De tweede vragenlijst werd gebruikt om het gebruik en de kennis die delaware heeft rond RPA te testen en te kijken welke RPA-tools delaware al gebruikt binnen het bedrijf. Deze werd gebruikt bij het interview met Sam Debruyn, aangezien hij deel uitmaakt van het Automation team wat instaat voor de ontwikkelingen van automatiseringen binnen delaware.

\subsection{Vragenlijst 1}
\label{subsec:vragenlijst-1}

Zoals hierboven vermeld werd deze vragenlijst vooral gebruikt om de algemene kennis van RPA binnen de teams te testen en te kijken in hoeverre dit al ingeburgerd is. Ook werd aan deze consultants gevraagd of ze bepaalde processen op het oog hadden die ze graag geautomatiseerd zien. Deze vragenlijst zag er als volgt uit:

\begin{itemize}
    \item Hoe familiair bent u met RPA?
    \item Van welke tools binnen RPA ben je op de hoogte? Zowel algemeen als binnen delaware?
    \item Welke criteria spelen volgens u een rol om een proces te selecteren voor automatisering?
    \item Zijn er bepaalde processen binnen de klantenservice van delaware die je graag geautomatiseerd zou zien?
    \item Moest een RPA bot zich aanbieden om je werk te kunnen versnellen door bijvoorbeeld gegevens op te halen of een form in te vullen, zou je deze gebruiken of zou je er twijfels bij hebben?
\end{itemize}

\subsection{Vragenlijst 2}
\label{subsec:vragenlijst-2}

Deze vragenlijst werd gesteld aan de technische consultant die al vaker met RPA heeft gewerkt. Via deze lijst werd achterhaald hoe ver delaware al staat met hun RPA en welke tools ze gebruiken. Deze vragenlijst zag er als volgt uit:

\begin{itemize}
    \item Wat is uw achtergrond in en ervaring met RPA technologie?
    \item Hoe verhouden de verschillende technologieën binnen delaware zich met betrekking tot het aantal projecten, maturiteit, etc.?
    \item Hoe wordt bij een project beslist wanneer een proces geautomatiseerd moet worden via RPA?
    \item Op basis van welke criteria wordt bepaald welke RPA-verdeler gebruikt/voorgesteld zal worden aan een klant?
    \item Waarom de keuze voor RPA boven andere manieren van automatiseren? Welke manieren nog overwogen?
    \item Voorkeur voor attended bot of unattended bot en waarom?
    \item Hoe ver denkt u dat de automatisering via RPA zal gaan binnen een paar jaar?
\end{itemize}

\section{Samenvatting interviews}
\label{sec:samenvatting-interviews}

Alvorens deze interviews af te nemen waren er 3 grote vragen die beantwoord moesten worden. In de volgende subsecties worden de antwoorden van de ondervraagden op deze vragen besproken.

\subsection{Huidige status RPA binnen delaware}
\label{subsec:huidige-status-rpa-binnen-delaware}

Delaware is een innovatief bedrijf, dus het mag als geen verrassing komen dat ze ook al werken met RPA-technologieën.
Het Automation team waar Sam Debruyn deel van uitmaakt werkt met meerdere RPA-tools, waaronder vooral UIPath, SAP BPA en Microsoft Power Automate. Dit is ook niet onverwacht, aangezien de grootste ecosystemen binnen delaware SAP en Microsoft zijn, en UIPath de marktleider is binnen RPA. Ook MuleSoft wordt in mindere mate gebruikt.
Deze automatiseringen zijn wel vooral gericht op de systemen van de klanten en worden door de medewerkers van delaware zelf minder gebruikt. Dit doordat het vaak de klant is die met RPA-voorstellen komt en niet andersom.
Bij de consultant uit het Finance team is er duidelijk zowel theoretische als praktische kennis aanwezig. Een aantal consultants zijn binnen het Finance team ook gecertificeerd voor UIPath. Dit komt vooral omdat RPA als eerste binnen dit team gebruikt werd, alvorens deze technologie naar het Automation Team werd overgebracht. Door deze overdracht wordt het nu binnen het SAP Finance team wel veel minder gebruikt. Dit komt opnieuw overeen met wat er in de literatuurstudie werd gevonden. Vele bedrijven beginnen met RPA binnen hun financiële sector, alvorens deze kennis uit te breiden naar andere bedrijfstakken en teams.
Maar RPA is niet binnen alle teams van delaware even gekend. Dit wordt duidelijk wanneer we de antwoorden van de consultants uit het SAP Development Team bekijken. Hier is er theoretische kennis aanwezig, maar amper praktische ervaring. Dit komt door het feit dat een aantal consultants binnen dit team het SAP BPA certificaat gevolgd hebben, maar deze kennis nog niet binnen projecten hebben gebruikt. De bekendste RPA tool binnen dit team is dus wel duidelijk SAP BPA, wat opnieuw geen verassing mag zijn aangezien de ondervraagden het meest in dit ecosysteem werken.

\subsection{Automatisering van processen binnen delaware}
\label{subsec:automatisering-van-processen-binnen-delaware}

Bij dit onderdeel wordt er een onderscheid gemaakt tussen de ondervraagden. De vragen aan Sam Debruyn waren vooral gericht op het proces dat doorlopen wordt bij een RPA-project, zoals het beslissen van de te automatiseren processen en het kiezen van de juiste verdeler, terwijl bij de andere ondervraagden eerder specifieke processen werden gezocht die hun werkdagen zou kunnen vergemakkelijken.

\subsubsection{Selectieproces binnen het Automation team}
\label{subsubsec:selectieproces-binnen-het-automation-team}

Bij de vragen rond het selectieproces voor RPA binnen delaware zijn er 3 verschillende onderwerpen voorgelegd aan Sam Debruyn.

Als eerste werd bevraagd wanneer er bij een project beslist wordt om een proces te automatiseren. Het antwoord op deze vraag was hetzelfde als wat er bevonden is binnen de literatuurstudie. De grootste factors binnen delaware zijn de frequentie waarmee een proces uitgevoerd wordt, of er een API beschikbaar is en of er meerdere systemen in een proces voorkomen. Er wordt ook regelmatig gekeken of er niet al een automatisering bestaat voor een bepaald proces. Vaak is het ook de klant die RPA voorstelt binnen een bepaald project omdat ze hier iets rond opgevangen hebben en deze innovatieve technologie willen integreren binnen hun bedrijf.

Als tweede werd bevraagd wanneer RPA de voorkeur krijgt boven andere manieren van automatiseren. Ook hier kwamen dezelfde antwoorden naar boven als binnen de literatuurstudie. Vooral het snellere resultaat en de afwezigheid van mogelijkheden binnen andere technieken zijn grote drijfveren. Er werd ook vermeld dat het feit dat RPA op het systeem werkt in plaats van in het systeem een belangrijke factor is bij de keuze voor RPA.

Als laatste werd bevraagd hoe er gekozen wordt welke RPA-technologie de voorkeur krijgt bij een project. Als eerste wordt er gekeken naar de complexiteit van het proces en de applicaties die binnen het proces gebruikt worden. Hiernaast wordt er natuurlijk geluisterd naar de wensen van de klant. Hier spelen de licentiekosten natuurlijk van groot belang en wordt er vaak gekeken of een klant niet al een ecosysteem heeft waar RPA makkelijker en goedkoper beschikbaar is, zoals credits bij SAP of Office 365. 

\subsubsection{Selectieproces buiten het Automation team}
\label{subsubsec:selectieproces-buiten-het-automation-team}

Aan de andere ondervraagden werden praktijkgerichtere vragen gesteld.

Als eerste werd gevraagd hoe groot hun huidige kennis was over RPA en van welke verdelers ze op de hoogte waren. Hier verschillen de antwoorden opnieuw per team. In het SAP Finance Team zijn er meerdere consultants met kennis van verschillende RPA-verdelers, zoals UIPath en Mulesoft.
De ondervraagden binnen het SAP Development Team hadden duidelijk ook kennis over het concept RPA, maar dit bleef eerder theoretisch. Het is wel duidelijk dat de kennis van verdelers zich binnen dit team beperkt tot SAP BPA, maar dit was te verwachten aangezien de ondervraagden SAP consultants zijn.

Ten tweede werd bevraagd welke processen ze in hun dagelijkse werkdag zelf zouden opgeven als automatiseerbaar via RPA. Opnieuw wisten de ondervraagden processen aan te geven die repetitief van aard zijn en waarbij vaak meerdere systemen gebruikt worden. Er werd ook gesproken over processen die een snelle ROI hebben en dus een 'quick win' zijn. De volledige lijst van processen die door de ondervraagden werd opgegeven is te vinden in sectie \ref{subsubsec:opties-voor-rpa-automatisering}, waar ze uitvoeriger besproken worden.

Als laatste werd gepolst naar hun beeld ten opzichte van RPA en of ze deze bots zouden gebruiken en vertrouwen moesten deze zich aanbieden. Hier zat er wat variantie op de antwoorden. Sommige ondervraagden zouden de bots direct gebruiken aangezien ze nieuwe, goedgekeurde technologieën meteen zouden aanvaarden als dit hun werk productiever maakt, terwijl anderen eerder wat terughoudender reageerden en eerst zelf hun onderzoek zouden doen naar de bot. Een handleiding over wat de bot precies doet en hoe je hem kan gebruiken lijkt wel een must.

\subsubsection{Opties voor RPA-automatisering}
\label{subsubsec:opties-voor-rpa-automatisering}

Uit de ondervragingen kwamen verschillende processen naar boven die volgens de ondervraagden baat zouden hebben bij een RPA-automatisering.
Deze processen zijn:

\begin{itemize}
    \item Lijst van Transport Requests overzetten tussen twee verschillende systemen.
    \item Ophalen van systeem-gegevens voor een bepaalde klant.
    \item Verkooporders goedkeuren.
    \item Klantenservice voor betalingen en orders.
    \item Bestanden inlezen en automatisch soorteren aan de hand van AI en ML binnen RPA.
    \item CI/CD pipeline opzetten.
    \item Aanmaken van een Statement of Work.
    \item Automatische Powerpoint maken op basis van een document.
\end{itemize}

Deze processen zijn vrij uiteenlopend, maar hebben allemaal de repetitieve aard gemeen. Bij het grootste deel zijn er ook meerdere systemen betrokken. Dit maakt duidelijk dat de ondervraagden goed begrijpen wat RPA is en de mogelijkheden hiervan wel al kunnen inschatten.
De processen worden in het volgende hoofdstuk verder onderzocht op frequentie, complexiteit en toegevoegde waarde om zo tot een keuze te komen voor de Proof of Concept.

