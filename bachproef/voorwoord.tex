%%=============================================================================
%% Voorwoord
%%=============================================================================

\chapter*{\IfLanguageName{dutch}{Woord vooraf}{Preface}}%
\label{ch:voorwoord}

%% TODO:
%% Het voorwoord is het enige deel van de bachelorproef waar je vanuit je
%% eigen standpunt (``ik-vorm'') mag schrijven. Je kan hier bv. motiveren
%% waarom jij het onderwerp wil bespreken.
%% Vergeet ook niet te bedanken wie je geholpen/gesteund/... heeft

% \lipsum[1-2]

Via delaware kwam ik in aanmerking met het concept van Robotic Process Automation. Deze nieuwe, innovatieve technologie sprak mij vrijwel direct aan, wat mij dan ook heeft geïnspireerd om deze bachelorproef er rond te schrijven.
Het feit dat ik via deze technologie de dagdagelijkse werking van de consultants zelf zou kunnen vergemakkelijken en hierbij ook nog eens de klant tevreden te stellen, leek mij reden genoeg om me hier in te verdiepen.
Ik heb dan ook heel veel bijgeleerd over dit onderwerp, en na deze bachelorproef heb ik nog meer vertrouwen in deze technologie en ben ik ervan overtuigd dat deze in de toekomst nog veel prominenter aanwezig zal zijn binnen de bedrijfswereld, zowel via kleine als grotere automatiseringen.
Ik hoop dat ik dit vertrouwen ook heb kunnen overbrengen naar de lezers van deze bachelorproef.

Via deze weg wil ik ook mijn promotor Leen Vuyge en co-promotor Jolien Lipkens bedanken voor het blijvende vertrouwen en de hulp bij deze bachelorproef.
Ook mijn ouders verdienen hier zeker een bedanking voor de blijvende steun en het nalezen van deze bachelorproef.