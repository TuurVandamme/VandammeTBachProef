%%=============================================================================
%% Conclusie
%%=============================================================================

\chapter{Conclusie}%
\label{ch:conclusie}

% TODO: Trek een duidelijke conclusie, in de vorm van een antwoord op de
% onderzoeksvra(a)g(en). Wat was jouw bijdrage aan het onderzoeksdomein en
% hoe biedt dit meerwaarde aan het vakgebied/doelgroep? 
% Reflecteer kritisch over het resultaat. In Engelse teksten wordt deze sectie
% ``Discussion'' genoemd. Had je deze uitkomst verwacht? Zijn er zaken die nog
% niet duidelijk zijn?
% Heeft het onderzoek geleid tot nieuwe vragen die uitnodigen tot verder 
%onderzoek?

Deze bachelorproef heeft aangetoond dat RPA wel degelijk een meerwaarde kan bieden voor bedrijven.
Wanneer we de resultaten van de Proof of Concept bekijken, zien we niet enkel een duidelijke tijdswinst en verhoging in efficiëntie, maar ook een positieve reactie van de betrokken werknemers.
Via deze resultaten werd de deelvraag 'Welke meerwaarde hebben deze automatiseringen?' direct beantwoord.
De deelvraag 'Welke taken kan Robotic Process Automation automatiseren?' werd deels door de literatuurstudie beantwoord. Processen met een hoge frequentie, een hoge repetitiviteit en weinig variantie bleken perfecte kandidaten voor RPA.
In de praktijk wisten de ondervraagde consultants zelf, zonder voorbereiding, met eigen kennis van hun dagelijkse processen meerder processen op te noemen die voor automatisering in aanmerking kwamen.

Deze twee deelvragen beantwoorden dan de centrale onderzoeksvraag 'Waar en hoe makkelijk kan Robotic Process Automation ingezet worden tijdens de klantenservice?'. Het is duidelijk dat RPA geen hoge drempel heeft om geïmplementeerd te worden binnen de dagelijkse werking van een bedrijf.
De 'waar' is moeilijker om een concreet antwoord op te geven, maar uit de literatuurstudie, de interviews en de Proof of Concept is gebleken dat RPA in elke bedrijfstak bruikbaar is. Het is dan natuurlijk wel belangrijk om de processen binnen deze bedrijfstak goed te analyseren en te toetsen tegen de criteria die in de literatuurstudie besproken werden.

Deze resultaten mogen geen verassing zijn, aangezien RPA een technologie is die nog steeds aan het winnen is aan populariteit, en dus duidelijk voor een reden.
Na dit onderzoek kunnen we besluiten dat RPA-automatiseringen een positieve invloed hebben op de werking van een bedrijf, waardoor er verwacht wordt dat delaware deze technologie zal blijven gebruiken en meerdere automatisering in gebruik zal nemen.