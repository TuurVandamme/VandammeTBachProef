%%=============================================================================
%% Methodologie
%%=============================================================================

\chapter{\IfLanguageName{dutch}{Methodologie}{Methodology}}%
\label{ch:methodologie}

Zoals uit de Hoofdstuk~\ref{ch:stand-van-zaken} is gebleken, is een correct proces kiezen en deze automatiseren niet altijd even eenvoudig. Daarom is het onderzoek binnen deze bachelorproef opgedeeld in 4 grote fases, die hieronder kort besproken worden.

\section{Literatuurstudie}
\label{sec:literatuurstudie}

De eerste fase van het onderzoek is de literatuurstudie. Hierin werd RPA in zijn gehele onderzocht. Er werd zowel gekeken naar het onstaan van RPA, de verschillende technologieën die RPA mogelijk maken en hoe RPA binnen een bedrijfscontext gebruikt kan worden.
Tenslotte werden de verschillende RPA-verdelers besproken.

\section{Interviews met consultants}
\label{sec:interviews-consultants}

Voor de tweede fase van het onderzoek werden 4 verschillende consultants binnen delaware ondervraagd. Met de kennis die in de eerste fase vergaard werd, werden de consultants gericht bevraagd over hun visie op en ervaring met RPA, zowel persoonlijk als binnen delaware. Uit deze interviews werden ook verschillende processen uit hun dagelijkse werkdag gehaald die later in het onderzoek vergeleken werden om zo tot een geschikt proces te komen voor de volgende fase binnen dit onderzoek.

\section{Proof of Concept}
\label{sec:proof-of-concept}

In de derde fase werd, met de kennis uit de vorige twee fases, een Proof of Concept uitgewerkt. Deze fase kan onderverdeeld worden in 3 sub-fases. Deze worden hieronder besproken.

\subsection{Keuze van het proces}
\label{subsec:keuze-proces}

Uit de tweede fase van het onderzoek kwamen verschillende processen die de consultants opgegeven hebben als processen die in aanmerking komen om geautomatiseerd te worden via RPA. Deze processen werden met elkaar vergeleken op verschillende criteria, waar ze per criteria een score op 3 kregen. Deze scores werden dan opgeteld en vergeleken, om zo het proces te kiezen wat volgens deze criteria het best geschikt is om uit te werken als een RPA-automatisering.

\subsection{Keuze van de RPA-software}
\label{subsec:keuze-software}

In de eerste fase van het onderzoek werden verschillende RPA-verdelers onderzocht en vergeleken. In de tweede fase werd dan gekeken naar de verschillende RPA-software waar delaware specifiek al ervaring mee had en hoe gekend deze waren bij de werknemers.
Met deze kennis en de kennis van het gekozen proces werd dan bekeken welke tool het best bij de Proof of Concept past.

\subsection{Uitwerking Proof of Concept}
\label{subsec:uitwerking-proof-of-concept}

Nu het proces en de RPA-software gekend waren, werd de Proof of Concept in de praktijk uitgewerkt. Hiervoor werden de verschilende stappen die in Hoofdstuk~\ref{ch:stand-van-zaken} besproken werden gevolgd.
De gecreëerde automatisering werd binnen deze fase, samen met een consultant, vergeleken met de huidige manier van werken. Hieruit werd dan geconcludeerd of de automatisering wel degelijk een meerwaarde biedt voor de consultant.

\section{Conclusie}
\label{sec:conclusie}

In deze laatste fase werd de conclusie getrokken uit de bevindingen uit de vorige fases. Vanuit deze conclusies werden de onderzoeksvraag en de deelvragen beantwoord. Hierna werd ook de toekomst van RPA binnen delaware besproken.