\chapter{\IfLanguageName{dutch}{Stand van zaken}{State of the art}}%
\label{ch:stand-van-zaken}

Robotic Process Automation (RPA) is de laatste jaren opgekomen als een van de belangrijkste nieuwe technologieën op het vlak van automatiseringen en digitale transformatie. RPA maakt gebruik van software 'robots' of 'bots' om repetitieve, tijdrovende en foutgevoelige taken te automatiseren. Hierdoor kunnen bedrijven hun processen efficiënter maken, kosten verlagen en productiviteit verhogen. 
RPA zorgt er ook voor dat werknemers zich niet meer hoeven te focussen op deze taken, waardoor ze zich, volgens het onderzoek van \textcite{ZalewskaTurzynska2022} meer kunnen focussen op het creëren van meerwaarde en zich kunnen focussen op inovatievere taken. Omdat RPA meer blijkt te zijn dan een nieuw buzzwoord en hierdoor in verschillende sectoren aan populariteit blijft winnen, is het een onderwerp waar er recent veel belangstelling aan wordt gehecht.
In dit onderdeel wordt onderzocht wat RPA precies is, hoe het is ontstaan en wat de toekomst zal brengen. Hiernaast wordt RPA bekeken vanuit de bedrijfscontext, en als laatste worden een aantal verdelers van deze software besproken.

\section{Robotic Process Automation}

Robotic Process Automation is nog steeds een relatief nieuwe technologie. Het verschilt van andere automatiseringssoftware in de manier waarop het interageert met de applicaties. Waar klassieke automatiseringstools 'inside-out' werken en dus in de applicatie-software zelf worden ingewerkt, werkt RPA op een 'outside-in' manier. Dit wil zeggen dat een RPA-automatisering werkt op de presentatielaag van een applicatie. 
Hier voert de automatisering taken uit zoals een eindgebruiker deze zou uitvoeren, maar volgens \textcite{Bras2023} met een hogere snelheid en precisie. Doordat RPA op deze manier werkt is er een minimale aanpassing nodig om een RPA-automatisering binnen een systeem te implementeren \autocite{Ivancic2019}.
Deze manier van interageren met de presentatielaag wordt bereikt door middel van agents die de software 'bots' kunnen aansturen. Een agent is een hardware-systeem (desktop, laptop, virtual machine, etc.) waarop de RPA-software is geïnstalleerd. Wanneer de RPA-software deze agent in werking zet, zal deze agent een software 'robot' starten die vervolgens het systeem overneemt en de automatisering uitvoert. 
Deze automatisering is dan in staat om een eindgebruiker van het systeem na te bootsen en via toetsaanslagen en muisbewegingen verschillende taken uit te voeren. Wanneer deze robot de automatisering heeft uitgevoerd, gaat hij terug in een soort slaapstand waardoor de gebruiker weer volledige controle krijgt over het hardware-systeem en wacht de agent op een nieuwe taak.
Volgens \textcite{Ivancic2019} bestaan er bij RPA twee verschillende soorten robots, attended en unattended robots. Attended bots zijn software robots die automatiseringen uitvoeren samen met een gebruiker. De robot voert bepaalde taken uit, maar geeft informatie terug aan een gebruiker die deze output controleerd, kan gebruiken in andere processen, etc. De unattended robots daarentegen, werken volledig autonoom en kunnen hun taak uitvoeren zonder tussenkomst van een gebruiker. 
Dit zijn vaak complexere taken met een grote hoeveelheid data die zeer repetitief verwerkt moet worden en waar de gebruiker geen meerwaarde van ondervindt om deze te controleren. \textcite{Axmann2022} voegt hier nog de hybrid bots aan toe, die een combinatie is van de twee andere soorten bots.

\section{Het ontstaan van Robotic Process Automation}
\label{sec:ontstaan-van-rpa}

\subsection{De begindagen van automatisering}
\label{subsec:begindagen-van-automatisering}

Het idee om repetitieve, handmatige taken te automatiseren is niet nieuw.
Automatisering maakt al decennialang deel uit van de industrie. Sinds de industriële revolutie zijn vele taken die vroeger handmatig werden uitgevoerd, vervangen door machines en robots. Denk hierbij bijvoorbeeld aan de lopende band-productie. Het toepassen van automatiseringstechnieken in bedrijfsprocessen en administratie is echter een recentere ontwikkeling, die te danken is aan deze eerste successen binnen de industrie.

\subsection{Het begin van Robotic Process Automation}
\label{subsec:begin-van-rpa}

Het succes van automatiseringen in de industrie heeft er dus voor gezorgd dat bedrijven ook in andere bedrijfsprocessen onderzoek zijn gestart om ook hier deze concepten toe te kunnen passen. Eind jaren 1990 werden de eerste software robots ontwikkeld die routinetaken in de productieomgeving konden automatiseren. Deze software robots waren echter nog niet van de omvang zoals vandaag, maar lijken volgens \textcite{ZalewskaTurzynska2022} meer op de huidige 'macros' die gekend zijn vanuit Microsoft Excel. 
Deze software robots konden repetitieve taken uitvoeren in 1 applicatie, maar waren niet in staat om verschillende applicaties aan te spreken. Maar zelfs bij het automatiseren van deze eenvoudige, repetitieve basistaken, zoals gegevensinvoer en basisprocesbeheer, werd automatisering toch al snel gezien als een waardevol hulpmiddel om de efficiëntie te verhogen en kosten te verlagen.

Sinds midden 2000 werd deze automatisering door verschillende bedrijven uitgebreid naar de wat nu gekend is als het begin van Robotic Process Automation. Hierdoor werd het mogelijk om automatiseringen handelingen te laten uitvoeren in verschillende applicaties \autocite{Fluss2020}, waardoor de mogelijkheden van deze automatiseringen snel de lucht in gingen.

Hierdoor begonnen bedrijven de potentie van RPA in te zien. Maar het is pas later, rond 2010, dat RPA echt aan populariteit begon te winnen.

\subsection{Robotic Process Automation als mainstream technologie}
\label{subsec:rpa-als-mainstream-technologie}

Zoals hierboven vermeld nam in 2010 het gebruik van RPA-technologieën aanzienlijk toe. Grote bedrijven, die werken met een complex IT-landschap die over de jaren heen bleef groeien, werden zich bewust van de potentiële kostenbesparingen en verbeterde efficiëntie die deze nieuwe technologie bood. De ontwikkeling van cloudgebaseerde RPA-software maakte de technologie ook veel toegankelijker aangezien er geen grote investeringen in een uitgebreide IT-infrastructuur meer nodig was.

Sinds 2010 is de RPA-markt elk jaar blijven toenemen. Volgens \textcite{Jiles2020} beginnen bedrijven vaak met RPA in de financiële afdeling, maar dit wordt meestal snel uitgebreid naar de verschillende bedrijfstakken. Door de vele interesse in de RPA technologie zijn de verschillende verdelers van deze software ook gegroeid, waardoor hun software steeds meer functionaliteiten aanbiedt en de mogelijkheden van RPA steeds groter worden. Ook de opkomst van nieuwe technologieën, zoals artificiële intelligentie (AI) en machinaal leren (ML) zorgen ervoor dat RPA steeds meer mogelijkheden biedt.

\subsection{De toekomst van Robotic Process Automation}
\label{subsec:toekomst-van-rpa}

De oorsprong van RPA gaat terug tot eind jaren negentig, toen de eerste software-automatiseringen to stand kwamen. In de loop der tijden is RPA geëvolueerd van een eenvoudige manier van automatiseren tot een volwaardige, mainstream en multifunctionele technologie waarmee bedrijven verschillende end-to-end bedrijfsprocessen kunnen automatiseren.

De voorbije jaren is de interesse in RPA duidelijk nog steeds toegenomen, aangezien RPA volgens \textcite{Laxmikant2023} één van de top 10 technologische trends is voor de komende 5 tot 10 jaar. Nieuwe technologieën zoals AI en ML, die steeds meer worden toegepast binnen de RPA-software, zorgen opnieuw voor extra mogelijkheden binnen de RPA-technologie. Deze mogelijkheden zullen er voor zorgen dat RPA nog complexere taken zal kunnen uitvoeren, waardoor de mogelijkheden voor RPA opnieuw zullen toenemen.

De toekomst voor RPA ziet er veelbelovend uit, wat voor vele bedrijven de trigger is om deze technologie te integreren binnen hun processen.

\section{Onderliggende technologieën}
\label{sec:onderliggende-technologieën}

Zoals hierboven besproken is RPA vrij recent ontstaan. Automatiseringen zelf bestaan al een heel stuk langer. De technologieën die RPA mogelijk maken zijn dus ook niet nieuw meer. Hieronder worden de belangrijkste bouwblokken die RPA mogelijk maken besproken.

\subsection{Presentatielaag (GUI)}
\label{subsec:presentatielaag}

RPA werkt op een 'outside-in' manier, wat het onderscheid van klassieke auotmatiserings-technieken. RPA gebruikt de presentatielaag zoals een normale gebruiker deze zou gebruiken. Onder presentatielaag wordt hier de laag van een applicatie die een gebruiker kan zien en waarmee hij kan interageren via toetsaanslagen en muisbewegingen, verstaan \autocite{ZalewskaTurzynska2022}.

\subsection{Screen Scraping}
\label{subsec:screen-scraping}

Screen scraping is een techniek die wordt gebruikt om de presentatielaag van een applicatie uit te lezen, waardoor deze gegevens gebruikt kunnen worden in een andere applicatie \autocite{Spencer2018}. Deze techniek wordt ook door RPA gebruikt, maar op een gesofisticeerdere manier. Oudere screen scraping software was niet in staat om elementen van een applicatie te herkennen, maar gebruikte vooral de relatieve positie van elementen om deze te kunnen gebruiken. 
Dit gaf natuurlijk problemen wanneer de presentatielaag aangepast werd, aangezien de relatieve positie van de elementen dan ook aangepast werd. Hierdoor werd het dus moeilijk om robuuste automatiseringen te maken die niet snel zullen falen. RPA werkt niet met de relatieve positie, maar 'herkent' de elementen waarmee het moet interageren. \textcite{Asquith2019} spreken over de identificatie van de elementen van een bepaalde pagina, waardoor RPA-software meer herkenpunten binnen de applicatie heeft en hierdoor dus minder afhankelijk is van de applicatie-layout en dus ook foutbestendiger kan werken.

\subsection{Optical Character Recognition (OCR)}
\label{subsec:ocr}

Bij sommige automatiseringen is het ook nodig om verschillende documenten te kunnen uitlezen. OCR is een technologie die het mogelijk maakt om tekst uit verschillende documenten te kunnen lezen. Denk hierbij aan e-mails van klanten, facturen, etc. Dit maakt het mogelijk voor RPA-automatiseringen om ook deze gegevens uit te lezen en te kunnen gebruiken in andere applicaties \autocite{ZalewskaTurzynska2022}.

\subsection{Application Programming Interface (API)}
\label{subsec:api}

Een extra voordeel van RPA-software is het feit dat het ook gebruik kan maken van API-calls om data te verzamelen. API-calls maken het mogelijk om verzoeken naar een systeem te versturen, waarna dit systeem dan bepaalde data terugstuurt. Dit antwoord wordt vaak teruggestuurd in het JSON-formaat. Dit vergroot de mogelijkheden om nog meer systemen samen te laten werken, op een snellere en robuustere manier. Vooral bij het automatiseren van processen die veel data nodig hebben om taken uit te voeren, is deze extra mogelijkheid volgens \textcite{Hofmann2020} een groot pluspunt in vergelijking met de normale, menselijke gebruikers.
Een API-call ziet er als volgt uit:
\begin{verbatim}
    curl 'https://northwind.netcore.io/customers/ALFKI/orders.json' \
      -H 'authority: northwind.netcore.io' \
      -H 'accept: text/html,application/xhtml+xml,application/xml' \
      -H 'accept-language: en-US,en;q=0.9,nl-NL;q=0.8,nl;q=0.7' \
      -H 'cache-control: no-cache' \
      -H 'pragma: no-cache' \
      -H 'referer: https://northwind.netcore.io/' \
      -H 'sec-ch-ua: "Chromium";v="120", "Google Chrome";v="120"' \
      --compressed
\end{verbatim}
Zoals hierboven te zien is, gebruikt de API-call dus een specifieke URL om een bepaald systeem aan te spreken. Daarnaast worden ook verschillende parameteters meegegeven, zoals de taal, de browser, het verwachte formaat van het antwoord, etc. Deze parameters kunnen verschillen van API tot API, waardoor het belangrijk is om goed te weten hoe de API die aangesproken wordt werkt en welke data deze verwacht.

Het antwoord op een API-call ziet er dan als volgt uit:
\begin{verbatim}
    {
      "results": [
        {
          "order": {
            "id": 10643,
            "customerId": "ALFKI",
            "employeeId": 6,
            "orderDate": "/Date(872467200000-0000)/",
            "requiredDate": "/Date(874886400000-0000)/",
            "shippedDate": "/Date(873158400000-0000)/",
            "shipVia": 1,
            "freight": 29.46,
            "shipName": "Alfreds Futterkiste",
            "shipAddress": "Obere Str. 57",
            "shipCity": "Berlin",
            "shipPostalCode": "12209",
            "shipCountry": "Germany"
          },
          "orderDetails": [
            {
              "orderId": 10643,
              "productId": 28,
              "unitPrice": 45.6,
              "quantity": 15,
              "discount": 0.25
            }
          ]
        }
      ]
    }
\end{verbatim}
Een API-call heeft dus op een snelle manier heel wat data terug. Deze data is logisch gestructureerd, maar niet altijd even overzichtelijk voor een menselijke gebruiker. Dit maakt API-calls dus aantrekkelijker om te gebruiken binnen RPA-automatiseringen, aangezien deze geen probleem hebben met het verwerken van deze informatie en dit op een veel snellere manier kunnen interpreteren.


\subsection{Data Extraction}
\label{subsec:data-extraction}

In de vorige sectie werd data afkomstig van API-calls besproken, maar RPA-software kan ook moeiteloos data uit andere bronnen halen, zoals Microsoft Excel, PDF, e-mails, etc. \autocite{Andrade2022}. Opnieuw is dit een voordeel ten opzichte van menselijke gebruikers, aangezien deze data vaak in grote pakken beschikbaar is en moeilijker leesbaar en bruikbaar is voor menselijke gebruikers.

\subsection{Artificiële Intelligentie (AI) en Machinaal Leren (ML)}
\label{subsec:ai-en-ml}

In subsectie~\ref{subsec:rpa-als-mainstream-technologie} zijn AI en ML al eens vermeld als technologieën die RPA gunstig kunnen beïnvloeden. Deze technologieën worden vaker en vaker ingezet bij RPA-software, waardoor de robots complexere taken kunnen behandelen, die meer intelligentie zullen vereisen. Denk hierbij aan het herkennen van afbeeldingen, teksten en e-mails begrijpen, etc. RPA-software zal hierdoor ook kunnen leren van zijn eigen fouten, waardoor het mogelijk zal zijn om robuustere automatiseringen te maken, die zichzelf constant kunnen verbeteren \autocite{Taulli2020}.

\section{Robotic Process Automation in de bedrijfswereld}
\label{sec:rpa-in-de-bedrijfswereld}

RPA is natuurlijk niet de enige automatiseringstool op de markt. Andere voorbeelden zijn bijvoorbeeld BPMN, een volledige integratie binnen het systeem, etc. Deze opties zijn vaak duurder. Ook werken ze op de 'inside-out' manier, waardoor de onderliggende applicaties aangepast moeten worden, wat op zijn beurt dan zorgt voor langere implementatieperiodes en een grotere kans op complicaties.
Maar wanneer de voordelen van RPA op een rijtje gezet worden, is het niet verbazend dat het de laatste jaren zo aan populariteit heeft gewonnen, vooral in de bedrijfswereld. RPA biedt bedrijven de kans om verschillende processen te automatiseren, waardoor de efficiëntie van deze bedrijven de hoogte kan ingaan. Onderzoek leert ons dat niet elk proces even geschikt is om geautomatiseerd te worden en dat de juiste keuze van processen belangrijk is voor het succes van de implementatie van een RPA-automatisering binnen een bepaald bedrijf. Hieronder wordt RPA binnen een bedrijfsomgeving meer in detail besproken.

\subsection{Stroomlijnen van bedrijfsprocessen}
\label{subsec:stroomlijnen-van-bedrijfsprocessen}

Bedrijven zijn constant op zoek naar mogelijkheden om hun bedrijfsprocessen te verbeteren. Hierbij kunnen verhoogde snelheid, verhoogde efficiëntie en besparingen de drijfveren zijn \autocite{Axmann2022}. Zoals hierboven beschreven, lijkt RPA een ideale oplossing te zijn om deze doelen te bereiken. Elk bedrijf heeft verschillende repetitieve processen die gebaseerd zijn op vaste regels, die werken met een groot volume aan data en die zeer foutgevoelig zijn. Dit zijn nu net de processen die ideaal zijn om geautomatiseerd te worden. 
Een extra voordeel hiervan is het feit dat de werknemer zich kan bezighouden met taken die meer menselijk inzicht vragen en dus van grotere waarde zijn voor het bedrijf. Hierdoor stijgt niet alleen de toegevoegde waarde van deze werknemer, maar ook de voldoening die deze krijgt uit zijn of haar werk, wat op zich dan weer zorgt voor een hogere productiviteit \autocite{ZalewskaTurzynska2022}.

\subsection{Kostenbesparing}
\label{subsec:kostenbesparing}

Een ander pluspunt waardoor bedrijven vaak in de richting van RPA kijken is volgens \textcite{Fernandez2021} de relatief lage kost. Aangezien RPA werkt op de presentatielaag (outside-in) van een applicatie, hoeft deze applicatie zelf niet of amper aangepast worden, waardoor de downtime van een systeem minimaal is en de RPA-automatisering snel kan worden geïmplementeerd en in gebruik worden genomen \autocite{Asquith2019}. Dit is een groot voordeel ten opzichte van de klassiekere automatiseringsmogelijkheden, die vaak grotere investeringen vragen en een hogere implementatieduur hebben. 
Hierdoor heeft RPA een relatief korte terugverdientijd (ROI), wat voor bedrijven natuurlijk een groot voordeel is.
Door deze relatief lage kost is RPA niet alleen beschikbaar voor grote bedrijven en marktleiders, maar zien we dat middelgrote en kleine bedrijven ook steeds meer interesse tonen in deze technologie.

\section{Verloop van een RPA-project}
\label{sec:verloop-van-een-rpa-project}

In sectie~\ref{sec:rpa-in-de-bedrijfswereld} is al besproken dat RPA met relatief weinig bronnen een hoge ROI kan opleveren, maar daarvoor moet RPA natuurlijk wel op de correcte processen worden toegepast.
Een bedrijf bevat duizenden processen in zijn dagelijkse werking, dus in deze paragrafen wordt besproken welke stappen volgens \textcite{El-Gharib2023} genomen moeten worden om een RPA-project succesvol te implementeren.

\subsection{Analyse en proces-evaluatie}
\label{subsec:analyse-en-proces-evaluatie}

De eerste stap binnen het implementatie-proces van RPA is het analyseren van de verschillende processen binnen een bedrijf die in aanmerking komen om via RPA geautomatiseerd te worden \autocite{vanDerAalst2018}. Hierbij wordt gekeken naar de frequentie waarmee het proces voorkomt, de complexiteit en of de verschillende stappen binnen het proces voldoende gestandardiseerd zijn. Wanneer deze factoren tegen de verschillende processen getoetst zijn, kan per proces ingeschat worden hoeveel baat het bedrijf erbij zouden hebben om deze via RPA te automatiseren, en kan er een ruwe schatting gemaakt worden van de ROI. 
Hiervoor worden van de verschillende processen de genomen stappen grondig geanalyseerd, wordt er gekeken waar deze processen geoptimaliseerd kunnen worden en worden deze nadien in een flow gegoten. Deze flow kan dan geanalyseerd worden, waarna er kan geëvalueerd worden of het proces welk degelijk in aanmerking komt voor automatisering.
Na de evaluatie van het proces is het natuurlijk ook belangrijk om te kijken welke RPA-software het best past bij onze automatisering. Bij bedrijven is het vaak ook belangrijk om te bekijken welke middelen ze in huis hebben. Hierbij zijn licenties, kennis binnen het bedrijf, etc. van groot belang.

\subsection{Planning en design}
\label{subsec:planning-en-design}

Wanneer een proces geselecteerd en geanalyseerd is voor automatisering, kan er begonnen worden aan de planning en het ontwerp van de automatisering \autocite{Fernandez2021}. In de planningsstap wordt een gedetaileerde flow gemaakt van het volledige proces, waarin alle stappen en de verschillende inputs en outputs gedefinieerd worden. Wanneer deze flow op punt staat, kan de design fase beginnen. Hierbij wordt gekeken naar de verschillende stappen binnen deze flow en hoe deze best geautomatiseerd kunnen worden binnen de verschillende mogelijkheden van RPA. Wanneer er voor elke stap een oplossing gevonden is, kan er een compleet ontwerp gemaakt worden van de RPA-flow.

\subsection{Ontwikkeling}
\label{subsec:ontwikkeling}

In de vorige stap is er een volledig design gemaakt voor de RPA-automatisering. Deze kan nu door de ontwikkelaar gebruikt worden om de RPA robot te ontwikkelen binnen de gekozen RPA-software. Het is belangrijk dat de ontwikkelaar zich bewust is van het proces dat hij aan het automatiseren is, en niet te ver afdwaalt van de gecreëerde workflow.

\subsection{Testen en validatie}
\label{subsec:testen-en-validatie}

Wanneer de automatisering ontwikkeld is, is het belangrijk deze voldoende te testen op zowel correctheid als robuustheid \autocite{Liu2023}. Hierbij is het belangrijk dat zowel de 'happy flow' als de 'exception flows' getest worden. Onder happy flow wordt de flow verstaan die een automatisering uitvoert wanneer alles goed gaat. Een exception flow is de flow die de automatisering uitvoert wanneer een stap faalt. Bij de happy flow is het belangrijk te kijken of de stappen die de automatisering moet uitvoeren correct gebeurd zijn, en bij de exception flow is het belangrijk te kijken of de automatisering hier correct op reageert en de juiste informatie teruggeeft aan de gebruiker. Ook is het belangrijk dat de automatisering in de exception flow geen verkeerde informatie in een systeem heeft binnengebracht.
In een latere fase van het testen is het mogelijk de RPA-robot in te zetten bij een kleine groep gebruikers of een deel van de uit te voeren processen om te kijken hoe het werkt in de praktijk. Hierna kan bij de gebruikers feedback gevraagd worden, of kunnen de systemen gecontroleerd worden op correcte data.

\subsection{Inzet, onderhoud en monitoring}
\label{subsec:inzet-onderhoud-en-monitoring}

Wanneer de RPA-automatisering voldoende getest is kan deze ingezet worden in de volledige praktijk. \textcite{Lievanomartinez2022} beschrijven dat het hierbij belangrijk is dat de gebruikers voldoende opgeleid zijn om te weten hoe de RPA robot werkt, hoe ze hem kunnen gebruiken en welke taken deze overneemt. De robot zelf heeft natuurlijk de middelen en juiste authorisatie nodig om de vooropgestelde taken uit te kunnen voeren. Volgens \textcite{vanDerAalst2018} is het ook belangrijk om de RPA-robot te blijven onderhouden, en wanneer er aan de onderliggende systemen aanpassingen worden aangebracht, deze ook opnieuw te controleren op een correcte werking om desastreuze gevolgen te vermijden.
Maar zelfs wanneer er geen aanpassingen aan het systeem zijn gemaakt, is het belangrijk om de verschillende RPA robots te blijven monitoren. Hierdoor kan de performantie van de robots gecontroleerd worden en in een volgende stap mogelijks opnieuw verbeterd worden.

\section{Verschillende verdelers van Robotic Process Automation software}
\label{sec:verschillende-verdelers-van-rpa-software}

RPA is een relatief nieuwe, inovatieve technologie. Hierdoor zijn er veel bedrijven die hun eigen RPA-software aanbieden, waardoor het belangrijk is om de verschillende verdelers met elkaar te vergelijken. \textcite{Carter2023} bespreekt het Gartner RPA Magic Quadrant en welke bedrijven we moeten zien als marktleiders, visionairs, challengers en niche spelers. Hieronder worden de belangrijkste verdelers binnen hun relatieve categorie besproken.

\subsection{UIPath}
\label{subsec:uipath}

UIPath is een Roemeens bedrijf, opgericht in 2005. Het is de marktleider van 2023, waar het de trend van 2022 mee verderzet. Het richt zich vooral op end-to-end processen. Ook is het een van de oplossingen die het verste staat in het integreren van AI binnen hun gebied \autocite{GartnerUIPath2023}.

\subsubsection{Voordelen van UIPath}
\label{subsubsec:voordelen-van-uipath}

\begin{itemize}
    \item UIPath is een van de oudste en een van de meest geavanceerde RPA oplossingen op de markt.
    \item UIPath heeft verschillende AI toepassingen geïntegreerd binnen hun volledige platform \autocite{Liliana2022}.
    \item UIPath heeft verschillende tools beschikbaar die het vinden van automatiseerbare processen binnen een bedrijf makkelijker maken.
    \item UIPath heeft meerdere out-of-the-box automatiseringen die direct ingezet en gebruikt kunnen worden binnen een bedrijf.
\end{itemize}


\subsection{Blueprism}
\label{subsec:blueprism}

Blue Prism is een Brits bedrijf, opgericht in 2001. Het behoort tot de marktleiders van 2023. Blue Prism richt zich vooral op grotere bedrijven. Voor deze bedrijven heeft het ook verschillende connectoren gecreëerd voor veelgebruikte applicaties binnen de bedrijfswereld \autocite{GartnerBluePrism2023}. Ook biedt Blue Prism gebundelde services aan voor deze bedrijven, met meerdere robot licenties, een dashboard, etc. Blue Prism richt zich vooral op unattended end-to-end processen \autocite{Laxmikant2023}.

\subsubsection{Voordelen van Blueprism}
\label{subsubsec:voordelen-van-blueprism}

\begin{itemize}
    \item Goed geïntegreerde AI en ML toepassingen binnen hun product.
    \item Makkelijk te schalen, wat belangrijk is voor grotere bedrijven.
    \item Toepassingen voor het volledige proces van RPA-implementatie, van het zoeken naar processen tot het in gebruik nemen van de automatiseringen.
    \item Vergevorderede e-mail AI.
\end{itemize}

\subsection{Automation 360}
\label{subsec:automation-360}

Automation 360 is een product van het Amerikaanse bedrijf Automation Anywhere, opgericht in 2003. Het behoort tot de marktleiders van 2023. Automation 360 richt zich vooral op attended bots. Het werkt via een cloud-platform, waardoor het relatief goedkoop en makkelijk schaalbaar is \autocite{GartnerAutomationAnywhere2023}.

\subsubsection{Voordelen van Automation 360}
\label{subsubsec:voordelen-van-automation-360}

\begin{itemize}
    \item Focus op attended robots, die gebruikers helpen met hun dagelijkse taken.
    \item Via hun Process Discovery tool kunnen de processen die het meest baat zullen hebben van automatiseringen ontdekt worden.
    \item Zowel geschikt voor kleine als middelgrote bedrijven.
    \item Volledig cloud gericht.
\end{itemize}

\subsection{Power Automate}
\label{subsec:power-automate}

Power Automate is de RPA oplossing van Microsoft. Het behoort ook tot de marktleiders van 2023. Het heeft verschillende connectoren naar verschillende Microsoft producten, wat het aantrekkelijk maakt voor bedrijven die dit ecosysteem gebruiken. Het richt zich zowel op attended als unattended bots \autocite{GartnerMicrosoftPA2023}.

\subsubsection{Voordelen van Power Automate}
\label{subsubsec:voordelen-van-power-automate}

\begin{itemize}
    \item Makkelijke connectoren met verschillende Microsoft producten zoals Azure, Power BI, etc.
    \item Een groot aantal out-of-the-box automatiseringen en connectoren.
    \item Goede integratie van API-mogelijkheden.
    \item AI mogelijkheden binnen zowel de creatie als het gebruik van de RPA.
\end{itemize}

\subsection{Cyclone Robotics}
\label{subsec:cyclone-robotics}

Cyclone Robotics is een bedrijf uit China, opgericht in 2015. Het is de enige challenger van 2023. Het heeft een sterke focus op bedrijfsbrede oplossingen en legt een grote focus op de AI-mogelijkheden binnen hun product. Het bedrijf kreeg de challenger status aangezien het zich op dit moment nog steeds volledig richt op de markt in Azië, maar er is zeker potentieel om uit te groeien tot een marktleider \autocite{GartnerCycloneRobotics2024}.

\subsubsection{Voordelen van Cyclone Robotics}
\label{subsubsec:voordelen-van-cyclone-robotics}

\begin{itemize}
    \item Bedrijfsbrede oplossingen.
    \item Sterke focus op AI en GPT-achtige oplossingen.
    \item Zeer marktgericht, maar op dit moment enkel in Azië.
\end{itemize}

\subsection{Pega}
\label{subsec:pegasystems}

Pega is de RPA oplossing van Pegasystems, een Amerikaans bedrijf opgericht in 1983. Het behoort tot de visionairs van 2023. Dit omdat het in 2022 een exceptionele groei heeft gekend. Het wordt gezien als een van de innovatievere oplossingen binnen RPA, maar het lijdt nog steeds onder zijn overdreven complexiteit en een gebrek aan gebruiksvriendelijkheid \autocite{GartnerPega2024}.

\subsubsection{Voordelen van Pega}

\begin{itemize}
    \item Focus op end-to-end automatiseringen.
    \item Goede integratie met andere Pega producten.
    \item Real-time controle over de bots.
    \item Zeer innovatieve oplossingen.
\end{itemize}

\subsection{MuleSoft}

MuleSoft is de RPA oplossing van Salesforce. Het behoort tot de visionairs van 2023. Salesforce heeft een grote naambekendheid en een grote product-portfolio, wat het een aantrekkelijke oplossing maakt voor bedrijven die al werken met het Salesforce ecosysteem. Alhoewel hun RPA-software zeker potentieel heeft, wordt het door Salesforce nog weinig gepromoot, waardoor de software zelf nog geen grote naambekendheid heeft \autocite{GartnerSalesforceMulesoft2024}. 

\subsubsection{Voordelen van MuleSoft}

\begin{itemize}
    \item Goede integratie met andere Salesforce producten.
    \item Gebruiksvriendelijk.
    \item Goede focus op AI binnen de automatiseringen.
    \item Duidelijk overzicht van de verschillende bots.
\end{itemize}

\subsection{SAP Build Process Automation}
\label{subsec:sap-build-process-automation}

SAP Build Process Automation is de RPA oplossing van SAP. Het behoort tot de visionairs van 2023. Het werkt makkelijk binnen het SAP ecosysteem, maar is zeker niet beperkt tot deze applicaties. SAP biedt vele paketten aan binnen voor software, en in veel van deze paketten is SAP BPA inbegrepen. Het richt zich zowel op attended als unattended bots \autocite{GartnerSAPBPA2023}.

\subsubsection{Voordelen van SAP Build Process Automation}
\label{subsubsec:voordelen-van-sap-build-process-automation}

\begin{itemize}
    \item Makkelijke integratie binnen een bestaand SAP Ecosysteem.
    \item Vlotte proces analyse via SAP Signavio.
    \item Relatief goedkoop wanneer een bedijf binnen het SAP Ecosysteem werkt.
    \item Goede AI en workflow mogelijkheden.
\end{itemize}

\subsection{Brity RPA}
\label{subsec:brity-rpa}

Brity RPA is de RPA oplossing van Samsung. Het is de belangrijkste niche speler van 2023. Als nieuwe speler binnen de RPA-markt heeft het nog niet veel naambekendheid, maar via het Samsung ecosysteem heeft het veel potentieel om te groeien. Het richt zich op dit moment vooral op de Aziatische markt \autocite{GartnerSamsungSDSBrityRPA2024}.

\subsubsection{Voordelen van Brity RPA}

\begin{itemize}
    \item Makkelijke integratie binnen het Samsung ecosysteem.
    \item Goede focus op AI en ML.
    \item Gebruiksvriendelijk.
\end{itemize}

\subsection{Overzicht verdelers}
\label{subsec:overzicht-verdelers}

In tabel \ref{tab:rpa-overzicht} worden de verschillende RPA-verdelers nog eens opgelijst met hun belangrijkste sterke punten.

\begin{longtable}{|p{3cm}|p{12cm}|}
    \hline
    \textbf{Verdeler} & \textbf{Positieve Punten} \\
    \hline
    UIPath & \begin{itemize}[left=0pt]
                \item Geavanceerde RPA-oplossingen.
                \item Goede AI-toepassingen.
                \item Proces-analyse tools.
                \item Out-of-the-box automatiseringen.
            \end{itemize} \\
    \hline
    Blue Prism & \begin{itemize}[left=0pt]
                    \item Goed geïntegreerde AI- en ML-toepassingen.
                    \item Makkelijk schaalbaar.
                    \item End-to-End processen.
                    \item Geavanceerde e-mail AI.
                \end{itemize} \\
    \hline
    Automation 360 & \begin{itemize}[left=0pt]
                        \item Focus op attended bots.
                        \item Process Discovery tool voor het ontdekken van geschikte processen.
                        \item Geschikt voor zowel kleine als middelgrote bedrijven.
                        \item Volledig cloud-gericht.
                    \end{itemize} \\
    \hline
    Power Automate & \begin{itemize}[left=0pt]
                        \item Makkelijke connectoren voor verschillende Microsoft producten.
                        \item Groot aantal out-of-the-box automatiseringen en connectoren.
                        \item Goede integratie van API-mogelijkheden.
                        \item AI-mogelijkheden binnen creatie en gebruik van RPA.
                    \end{itemize} \\
    \hline
    Cyclone Robotics & \begin{itemize}[left=0pt]
                            \item Bedrijfsbrede oplossingen.
                            \item Sterke focus op AI en GPT-achtige oplossingen.
                            \item Marktgericht, maar momenteel vooral in Azië.
                        \end{itemize} \\
    \hline
    Pega & \begin{itemize}[left=0pt]
                \item Focus op end-to-end automatiseringen.
                \item Goede integratie met andere Pega producten.
                \item Real-time controle over de bots.
                \item Zeer innovatieve oplossingen.
            \end{itemize} \\
    \hline
    MuleSoft & \begin{itemize}[left=0pt]
                    \item Goede integratie met andere Salesforce producten.
                    \item Gebruiksvriendelijk.
                    \item Focus op AI binnen automatiseringen.
                    \item Duidelijk overzicht van verschillende bots.
                \end{itemize} \\
    \hline
    SAP Build Process Automation & \begin{itemize}[left=0pt]
                                        \item Makkelijke integratie binnen een bestaand SAP Ecosysteem.
                                        \item Vlotte procesanalyse via SAP Signavio.
                                        \item Relatief goedkoop binnen het SAP Ecosysteem.
                                        \item Goede AI en workflow mogelijkheden.
                                    \end{itemize} \\
    \hline
    Brity RPA & \begin{itemize}[left=0pt]
                    \item Makkelijke integratie binnen het Samsung ecosysteem.
                    \item Goede focus op AI en ML.
                    \item Gebruiksvriendelijk.
                \end{itemize} \\
    \hline

    \caption{Overzicht van RPA-verdelers en hun positieve punten.}
    \label{tab:rpa-overzicht}
\end{longtable}

\section{Veiligheid}
\label{sec:veiligheid}

Een grote zorg bij het implementeren van nieuwe technologieën binnen een bedrijf is vaak de veiligheid. Dit is natuurlijk ook het geval bij RPA. In het onderzoek van \textcite{Liu2023} bijvoorbeeld wordt bij elke flow die onderzocht wordt ook de veiligheid van de gegevens onderzocht en dit scoort bij elke flow relatief hoog.
Toch is het belangrijk om verschillende stappen te ondernemen om de systemen veilig te houden. \textcite{Taulli2020} bespreekt in zijn handboek een aantal veiligheids risico's die we hieronder zullen bespreken.
RPA zelf is een veilige technologie, maar het is belangrijk om de juiste stappen te ondernemen zodat een veilig gebruik ervan kan gegarandeerd worden.

\subsection{Inloggegevens}
\label{subsec:inloggegevens}

Een eerste risico is het gebruik van inloggegevens. Aangezien RPA-automatiseringen werken op de presentatielaag van een applicatie heeft het bepaalde inloggegevens nodig om toegang te krijgen tot deze applicatie. Deze inloggegevens worden vaak hardgecodeerd.
Dit is natuurlijk een groot veiligheidsrisico, aangezien deze hardgecodeerde gegevens in de foute handen toegang zouden kunnen verlenen tot deze systemen. Het is daarom zeer belangrijk om deze gegevens op een veilige, dynamische manier op te slaan.

\subsection{Systeemtoegang}
\label{subsec:systeemtoegang}

Het is makkelijk om een RPA-automatisering toegang te geven tot het volledige systeem zodat het alle taken kan uitvoeren die het ooit zou moeten uitvoeren. Dit is natuurlijk bad practice aangezien de automatisering enkel toegang zou moeten hebben tot de delen die het echt nodig heeft.
Het afschermen van bepaalde delen en het geven van beperkte toegang is dus ook een belangrijke stap in het beveiligen van de systemen.

\subsection{Beheer}
\label{subsec:beheer}

Een ander belangrijk punt is het beheer van de RPA-automatiseringen. Dit begint direct bij de beslissing van een RPA-verdeler. Wanneer een RPA-oplossing wordt gekozen is het belangrijk om te kijken naar het niveau van beveiliging die deze oplossing met zich meebrengt. Hiernaast is het belangrijk om als bedrijf een soort veiligheidsframework op te bouwen waarin een duidelijke definitie van rollen, verantwoordelijkheden en best practices worden opgenomen.

\subsection{Registreren van acties}
\label{subsec:registreren-van-acties}

Het is natuurlijk mogelijk dat een automatisering een foute actie uitvoert of de foute gegevens gebruikt voor een bepaalde actie. Het is daarom belangrijk dat alle acties die een RPA robot onderneemt geregistreerd worden en dat deze logs geraadpleegd kunnen worden. Dit vergemakkelijkt het onderzoek naar de oorzaak van automatiseringsproblemen en hierdoor kan ook bekeken worden welke acties de robot wel en niet heeft uitgevoerd.

\section{Waarom RPA}
\label{sec:waarom-rpa}

In de vorige secties werden de verschillende aspecten van RPA besproken. Hieruit is gebleken dat RPA niet zomaar een nieuwe technologie is, maar voor vele bedrijven een oplossing kan bieden voor verschillende langdurige processen. Maar waarom noemt \textcite{Fluss2020} RPA als een van de vijf technologische trends met een veelbelovende toekomst binnen de klantenservice? Hieronder wordt nog eens een overzicht gegeven van de voordelen van RPA.

\begin{itemize}
    \item Kostenbesparing.
    \item Korte ontwikkelingsduur.
    \item Outside-in aanpak, geen aanpassingen aan oude systemen.
    \item Beperking van menselijke fouten.
    \item Verhoogde efficiëntie.
    \item Grotere voldoening voor werknemers.
    \item Verhoogde productiviteit.
\end{itemize}

Als deze voordelen worden samengevat, is de interesse voor RPA niet verbazend. Dit is ook te zien in de wereldwijde cijfers. De RPA markt heeft in het jaar 2022 een groei gekend van 22 procent, wat ver boven de 11 procent staat van het gemiddelde software segment \autocite{Carter2023}.
RPA blijft dus winnen aan populariteit, wat een veelbelovende toekomst voorspelt voor deze technologie.